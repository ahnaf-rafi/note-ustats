\documentclass[12pt]{article}
\usepackage[T1]{fontenc}
\usepackage[utf8]{inputenc}

%%%%%%%%%%%
% Margins %
%%%%%%%%%%%
\usepackage[letterpaper, portrait, margin=1in]{geometry}

%%%%%%%%%%%%%%%%
% Line Spacing %
%%%%%%%%%%%%%%%%
\usepackage{setspace}
\doublespacing

%%%%%%%%%%%%%%%%%%%%%%%%%%%%%%%%
% References and links in text %
%%%%%%%%%%%%%%%%%%%%%%%%%%%%%%%%
\usepackage[bookmarks=true, hidelinks]{hyperref}
\PassOptionsToPackage{unicode}{hyperref}
\PassOptionsToPackage{naturalnames}{hyperref}
\hypersetup{
pdfauthor={Ahnaf Rafi},
colorlinks=true,
linkcolor=blue,
urlcolor=blue,
citecolor=blue
}

%%%%%%%%%%%%%%%%%
% Control lists %
%%%%%%%%%%%%%%%%%
\usepackage{enumitem}

%%%%%%%%%%
% Tables %
%%%%%%%%%%
% \usepackage{float}
% \usepackage{booktabs}
% \usepackage{multirow}
% \usepackage{multicol}

%%%%%%%%
% Math %
%%%%%%%%
\usepackage{amsmath}
\usepackage{amsfonts}
\usepackage{amssymb}
\usepackage{amsthm}
\usepackage{bm}
\usepackage{mathrsfs}

% Algorithms
% \usepackage[linesnumbered,ruled,vlined]{algorithm2e}

% Must be loaded after amsmath, but before new theorem defs.
\usepackage[nameinlink]{cleveref}

% Extra operators and symbols
% \DeclareMathOperator*{\argmax}{argmax}
\DeclareMathOperator*{\argmin}{argmin}
% \DeclareMathOperator*{\Leb}{Leb}

% Independence in Probability Theory and Statistics.
\newcommand{\indep}{\mathrel{\perp \mspace{-11mu} \perp}}

% Probability with normal subscripts and superscripts
\def\Pr{\mathop{\rm Pr}\nolimits}

% \crefname{subsubsubappendix}{Appendix}{appendix}

% Display breaks in aligned equations
\allowdisplaybreaks

\numberwithin{equation}{section}

% Theorem environments
\theoremstyle{definition}
% \newtheorem{definition}{Definition}

% \newtheorem{assumption}{Assumption}[section]
% \crefname{assumption}{assumption}{assumptions}

% \newtheorem{asm}{Assumption}
% \crefname{asm}{assumption}{assumptions}

% \newtheorem{remark}{Remark}[section]

% \newtheorem{example}{Example}[section]

\newtheorem{notation}{Notation}[section]

\theoremstyle{plain}
\newtheorem{theorem}{Theorem}[section]

\newtheorem{lemma}{Lemma}[section]
% \crefname{lemma}{lemma}{lemmas}

% \newtheorem{corollary}{Corollary}

\usepackage{etaremune}

\usepackage{calc}
\usepackage{crossreftools}

%%%%%%%%%%%%%%%%
% Bibliography %
%%%%%%%%%%%%%%%%
\usepackage[
authordate,
sorting=nyt,
backend=biber,
maxcitenames=3,
natbib
% doi=false,
% isbn=false,
% url=false,
% eprint=false
]{biblatex-chicago}

\addbibresource{note-ustats.bib}

\title{Some notes on U-Statistics}
\author{Ahnaf Rafi}
\date{\today}

\begin{document}

\maketitle

\tableofcontents

\section{Definitions of U-Statistics and V-Statistics}

Throughout, all randomness is defined on a probability space \((\Omega,
\mathscr{F}, \Pr)\) with \(\mathrm{E} [\cdot]\) and \(\mathrm{Var} [\cdot]\)
denoting expectation and variance respectively against \(\Pr\).
Furthermore, \(\left\{ Z_{i} \right\}_{i = 1}^{n}\) is a sample of i.i.d. random
vectors, with \(F\) denoting their common distribution.
Finally \(\|\cdot\|\) denotes the Euclidean norm for vectors.

For \(n, m \in \mathbb{N}\) with \(n \geq m\), define
\begin{equation}
  \begin{split}
    \mathbb{N}_{n} =
    & \, \{1, \dots, n\} = \{j \in \mathbb{N} : j \leq n\}, \\
    \mathrm{Inj}_{n, m} =
    & \, \left\{ i \in \mathbb{N}_{n}^{m} : \left[ j, k \in
    \mathbb{N}_{m}, j \neq k \right] \implies i (j) \neq i (k)
    \right\}, \\
    \mathrm{NonInj}_{n, m} =
    & \, \mathbb{N}_{n}^{m} \setminus \mathrm{Inj}_{n, m} =
    \left\{ i \in \mathbb{N}_{n}^{m} : \exists j, k \in \mathbb{N}_{m},
    j \neq k \text{ such that } i (j) = i (k) \right\}.
  \end{split}
  \label{eqn--finite-domain-codomain-func-sets}
\end{equation}
In \eqref{eqn--finite-domain-codomain-func-sets}, \(\mathrm{Inj}\) stands for
``injective'' (i.e. one-to-one) and \(\mathrm{NonInj}\) stands for
``non-injective''.
The cardinality of \(\mathrm{Inj}_{n, m}\) is
\begin{equation}
  \gamma_{n, m} := \left| \mathrm{Inj}_{n, m} \right| = \prod_{k = 0}^{m - 1} (n
  - k).
  \label{eqn--gammanm}
\end{equation}

For a function \(\eta \left( x_{1}, \dots, x_{m} \right)\), the associated U-
and V-statistics are respectively:
\begin{equation}
  \begin{split}
    U_{n} (\eta) =
    & \, \frac{1}{\gamma_{n, m}} \sum_{i \in \mathrm{Inj}_{n, m}} \eta \left(
    Z_{i (1)}, \dots, Z_{i (m)} \right),
    \\
    V_{n} (\eta) =
    & \, \frac{1}{n^{m}} \sum_{i \in \mathbb{N}_{n}^{m}} \eta \left( Z_{i (1)},
    \dots, Z_{i (m)} \right).
  \end{split}
  \label{eqn--U-V-stat}
\end{equation}
The function \(\eta\) is called a \emph{kernel} and the number of arguments
\(m := m_{\eta}\), is called the \emph{order} of the kernel.
Though the order \(m\) is dependent on \(\eta\), we suppress this dependence for
brevity.
In addition, we hold \(m\) fixed throughout and impose the restriction \(n \geq
m\).

The kernel \(\eta\) is permutation symmetric iff
\begin{equation*}
  \begin{gathered}
    \text{for any permutation } \left\{ i (1), \dots, i (m) \right\} \text{ of }
    \{1, \dots, m\}, \\
    \eta \left( z_{i (1)}, \dots, z_{i (m)} \right) = \eta \left( z_{1},
    \dots, z_{m} \right).
  \end{gathered}
\end{equation*}
When \(\eta\) is permutation symmetric, the corresponding U-statistic in
\eqref{eqn--U-V-stat} becomes
\begin{equation*}
  U_{n} (\eta) = \binom{n}{m}^{- 1} \sum_{i \in \mathrm{Inc}_{n, m}} \eta
  \left( Z_{i (1)}, \dots, Z_{i (m)} \right),
\end{equation*}
where \(\mathrm{Inc}_{n, m} \subseteq \mathrm{Inj}_{n, m}\) is the collection of
strictly increasing index vectors:
\begin{equation}
  \mathrm{Inc}_{n, m} = \left\{ (i (1), \dots, i (m)) \in \mathbb{N}_{n}^{m} :
  \left[ j, l \in \mathbb{N}_{m}, j < l \right] \implies i (j) < i (l) \right\}.
  \label{eqn--monotone-indices}
\end{equation}

\section{First order sum-space projection of U-statistics}

We now consider the projection of the U-statistic \(U_{n} (\eta, F)\) onto the
sum-space for the \(n\) i.i.d. random variables \(Z_{1}, \dots, Z_{n}\) drawn
from \(F\).
Denote this sum space by
\begin{equation}
  \mathcal{S}_{n} := \mathcal{S}_{n} (F) := \left\{ \sum_{j = 1}^{n} g_{j}
  \left( Z_{j} \right) : \forall j \in \mathbb{N}_{n}, Z_{j} \sim F, g_{j}
  \text{ measurable}, \int g_{j}^{2} \mathrm{d} F < \infty \right\}.
  \label{eqn--sum-space}
\end{equation}
Henceforth, let \(m \geq 2\).
Denote
\begin{align}
  \Pi_{0} (\eta) :=
  & \, \Pi_{0} (\eta, F) := \mathrm{E}_{F^{m}} \left[ \eta
  \left( Z_{1}, \dots, Z_{m} \right) \right],
  \\
  \begin{split}
    \forall c \in \mathbb{N}_{m - 1}, \quad \Pi_{c} \left( z_{1}, \dots, z_{c};
    \eta \right) :=
    & \, \Pi_{c} \left( z_{1}, \dots, z_{c}; \eta, F \right) \\
    :=
    & \, \mathrm{E}_{F^{m}} \left[ \eta \left( Z_{1}, \dots, Z_{m} \right) |
    Z_{1} = z_{1}, \dots, Z_{c} = z_{c} \right] \\
    =
    & \, \int \eta \left( z_{1}, \dots, z_{c}, z_{c + 1}, \dots, z_{m} \right)
    F^{m - c} \left(
    \mathrm{d} z_{c + 1}, \dots, \mathrm{d} z_{m} \right),
  \end{split} \\
  \begin{split}
    \text{and} \quad \widehat{U}_{n} (\eta) :=
    & \, \widehat{U}_{n} (\eta, F)  \\
    :=
    & \, \Pi_{0} (\eta, F) + \frac{m}{n} \sum_{j = 1}^{n} \left\{ \Pi_{1}
    \left( Z_{j}; \eta, F \right) - \Pi_{0} (\eta, F) \right\} \\
    =
    & \, \Pi_{0} (\eta) + \frac{m}{n} \sum_{j = 1}^{n} \left\{ \Pi_{1} \left(
    Z_{j}; \eta \right) - \Pi_{0} (\eta) \right\} .
  \end{split}
  \label{eqn--U-stat-proj}
\end{align}
In all of the above expressions, the notation expresses the fact that though the
functions \(\Pi_{c}\) and \(\widehat{U}_{n}\) all depend on both \(\eta\) and
\(F\), we shall suppress dependence on \(F\).
We follow this scheme throughout, so that \(\mathrm{E} \equiv
\mathrm{E}_{F^{n}}\) throughout.

\begin{lemma}
\label{lem--U-stat-proj}
\(\widehat{U}_{n} (\eta)\) in \eqref{eqn--U-stat-proj} is the projection
of \(U_{n} (\eta)\) onto the sum space \(\mathcal{S}_{n}\) in
\eqref{eqn--sum-space}.
\end{lemma}

\begin{proof}[Proof of \Cref{lem--U-stat-proj}]
We apply \Cref{thm--proj-sum-space}.
Take any \(i \in \mathrm{Inj}_{n, m}\).
Then, for any \(j = 1, \dots, n\),
\begin{equation*}
  \mathrm{E} \left[ \eta \left( Z_{i (1)}, \dots, Z_{i (m)} \right)
  \middle| Z_{j} = z \right] =
  \begin{cases}
    \Pi_{1} (z; \eta) & \text{if } j \in
    \left\{ i (1), \dots, i (m) \right\}, \\
    \Pi_{0} (\eta) & \text{otherwise}.
  \end{cases}
\end{equation*}
For fixed \(j \in \{1, \dots, n\}\), consider averaging over \(i \in
\mathrm{Inj}_{n, m}\).
The first case happens \(m \cdot \gamma_{n - 1, m - 1}\) times.
To see this, given an `empty' \(m\)-vector \(i\), there are \(m\) possible
places to place \(j\) and \(\gamma_{n - 1, m - 1}\) ways to select the remaining
elements of \(i\).
Hence
\begin{align*}
  \mathrm{E} \left[ U_{n} (\eta) \middle| Z_{j} \right] =
  & \, \frac{1}{\gamma_{n, m}} \sum_{i \in \mathrm{Inj}_{n, m}} \mathrm{E}
  \left[ \eta \left( Z_{i (1)}, \dots, Z_{i (m)} \right) \middle| Z_{j} \right]
  \\
  =
  & \, m \frac{\gamma_{n - 1, m - 1}}{\gamma_{n, m}} \Pi_{1} \left( Z_{j}; \eta
  \right) + \left( 1 - m \frac{\gamma_{n - 1, m - 1}}{\gamma_{n, m}} \right)
  \Pi_{0} (\eta).
\end{align*}
Since \(\gamma_{n - 1, m - 1} / \gamma_{n, m} = 1 / n\),
\begin{equation*}
  \mathrm{E} \left[ U_{n} (\eta) \middle| Z_{j} \right] = \frac{m}{n}
  \Pi_{1} \left( Z_{j}; \eta \right) + \left( 1 - \frac{m}{n} \right) \Pi_{0}
  (\eta).
  % \label{eqn--U-stat-proj-single-j}
\end{equation*}
Apply \Cref{thm--proj-sum-space}; the projection of \(U_{n} (\eta)\) is
therefore
\begin{align*}
  & \mathrm{E} \left[ U_{n} (\eta) \right] + \sum_{j = 1}^{n}
  \left( \mathrm{E} \left[ U_{n} (\eta) \middle| Z_{j} \right] - \mathrm{E}
  \left[ U_{n} (\eta) \right] \right) \\
  =
  & \, \Pi_{0} (\eta) + \sum_{j = 1}^{n} \left( \frac{m}{n}
  \Pi_{1} \left( Z_{j}; \eta \right) + \left( 1 - \frac{m}{n}
  \right) \Pi_{0} (\eta) - \Pi_{0} (\eta) \right) \\
  =
  & \, \Pi_{0} (\eta) +  \frac{m}{n} \sum_{j = 1}^{n}
  \left\{ \Pi_{1} \left( Z_{j}; \eta \right) - \Pi_{0} (\eta)
  \right\} \\
  =
  & \, \widehat{U}_{n} (\eta),
\end{align*}
as desired.
\end{proof}

\section{\texorpdfstring{\(\mathscr{L}_{2}\)}{L2} distance between a U-statistic
and its first order sum-space projection under permutation symmetry}

In this section, we assume that the kernel \(\eta\) is permutation symmetric.
Hence, the relevant set of indices to work with is \(\mathrm{Inc}_{n, m}\)
in \eqref{eqn--monotone-indices}.
Consider the (squared) \(\mathscr{L}_{2}\) distance between a U-statistic and
its projection, \(\widehat{U}_{n} (\eta, F)\), onto the first order sum space
\(\mathcal{S}_{n}\) defined in
\eqref{eqn--sum-space}:
\begin{equation}
  \rho_{n} (\eta) := \rho_{n} (\eta, F) := \mathrm{E}_{F^{n}} \left[
  \left( U_{n} (\eta) - \widehat{U}_{n} (\eta, F) \right)^{2} \right].
  \label{eqn--L2-dist-proj}
\end{equation}
We can readily apply \Cref{thm--proj-L2-dist-as-dif-of-vars} since
\(\mathcal{S}_{n}\) contains the constant (non-stochastic) random
variables.
Hence, \eqref{eqn--L2-dist-proj} becomes
\begin{equation}
  \rho_{n} (\eta) = \mathrm{Var} \left[ U_{n} (\eta) \right] - \mathrm{Var}
  \left[ \widehat{U}_{n} (\eta) \right].
  \label{eqn--Ustat-proj-L2-dist-as-dif-of-vars}
\end{equation}
The variance of \(\widehat{U}_{n} (\eta)\) is easily characterized:
\begin{equation}
  \mathrm{Var} \left[ \widehat{U}_{n} (\eta) \right] = \frac{m^{2}}{n}
  \mathrm{Var} \left[ \Pi_{1} (Z; \eta) \right].
  \label{eqn--Uhat-var-1}
\end{equation}
Furthermore, we can always write
\begin{equation}
  \mathrm{Var} \left[ U_{n} (\eta) \right] = \binom{n}{m}^{- 2} \sum_{i \in
  \mathrm{Inc}_{n, m}} \sum_{i^{\prime} \in \mathrm{Inc}_{n, m}}
  \mathrm{Cov} \left( \eta \left( Z_{i (1)}, \dots, Z_{i (m)}, \right), \eta
  \left( Z_{i^{\prime} (1)}, \dots, Z_{i^{\prime} (m)} \right) \right).
  \label{eqn--Ustat-var-1}
\end{equation}

We now introduce a notation scheme that will help us to simplify
\eqref{eqn--Ustat-var-1} and will allow us to provide a unified treatment of
\eqref{eqn--Uhat-var-1} and \eqref{eqn--Ustat-var-1} under i.i.d. \(Z_{1},
\dots, Z_{n}\).
For \(0 \leq c \leq m\), denote
\begin{equation}
  \begin{split}
  \zeta_{c} (\eta) := \zeta_{c} (\eta, F) :=
  & \, \mathrm{Var}_{F^{c}} \left[ \Pi_{c} \left( Z_{1}, \dots,
  Z_{c}; \eta, F \right) \right] \\
  =
  & \, \mathrm{Var}_{F^{c}} \left[ \mathrm{E}_{F^{m}} \left[ \eta \left( Z_{1},
  \dots, Z_{m} \right) \middle| Z_{1}, \dots, Z_{c} \right] \right].
  \end{split}
  \label{eqn--zeta-c}
\end{equation}
It can be shown that \(\zeta_{c} (\eta)\) has the following alternative
representation:
\begin{equation}
  \zeta_{c} (\eta) =
  \mathrm{Cov} \left( \eta \left( Z_{1}, \dots, Z_{c}, Z_{1, c + 1}, \dots,
  Z_{1, m} \right), \eta \left( Z_{1}, \dots, Z_{c}, Z_{2, c + 1}, \dots, Z_{2,
  m} \right) \right),
  \label{eqn--zeta-c-cov-eta-c-common}
\end{equation}
where \(\left\{ Z_{j} \right\}_{j = 1}^{n}\), \(\left\{ Z_{1, j} \right\}_{j =
1}^{n}\) and \(\left\{ Z_{2, j} \right\}_{j = 1}^{n}\) are all mutually
independent random vectors all with distribution \(F\).
That is, \(\zeta_{c} (\eta)\) is the covariance of \(\eta \left( Z_{i (1)},
\dots, Z_{i (m)} \right)\) and \(\eta \left( Z_{i^{\prime} (1)}, \dots,
Z_{i^{\prime} (m)} \right)\) when \(i\) and \(i^{\prime}\) are strictly
increasing and have exactly \(c\) elements in common.
Of course for \(c = 0\), \(\zeta_{0} (\eta) = 0\) by independence of the
\(Z_{j}\)'s.

Revisiting \eqref{eqn--Uhat-var-1} using \eqref{eqn--zeta-c},
\begin{equation}
  \mathrm{Var} \left[ \widehat{U}_{n} (\eta) \right] = \frac{m^{2}}{n} \zeta_{1}
  (\eta).
  \label{eqn--Uhat-var}
\end{equation}
For \eqref{eqn--Ustat-var-1}, we can instead use
\eqref{eqn--zeta-c-cov-eta-c-common}.
Since there are \(\binom{n}{m} \binom{m}{c} \binom{n - m}{m - c}\) pairs \(i,
i^{\prime} \in \mathrm{Inc}_{n, m}\) with exactly \(c\) elements in common (see
\Cref{thm--picking-with-common-elements} for a formal statement and proof),
\begin{align*}
  \mathrm{Var} \left[ U_{n} (\eta) \right] =
    & \, \binom{n}{m}^{- 2} \sum_{c = 0}^{m} \binom{n}{m} \binom{m}{c} \binom{n
    - m}{m - c} \zeta_{c} (\eta) \\
    =
    & \, \binom{n}{m}^{- 2} \sum_{c = 1}^{m} \binom{n}{m} \binom{m}{c} \binom{n
    - m}{m - c} \zeta_{c} (\eta) \qquad [\text{by } \zeta_{0} (\eta) = 0],
\end{align*}
and so,
\begin{equation}
  \mathrm{Var} \left[ U_{n} (\eta) \right] = \sum_{c = 1}^{m} \binom{m}{c}
  \binom{n}{m}^{- 1} \binom{n - m}{m - c} \zeta_{c} (\eta).
  \label{eqn--Ustat-var}
\end{equation}
Combining \eqref{eqn--Ustat-proj-L2-dist-as-dif-of-vars}, \eqref{eqn--Uhat-var}
and \eqref{eqn--Ustat-var},
\begin{equation}
  \rho_{n} (\eta) = \sum_{c = 1}^{m} \binom{m}{c} \binom{n}{m}^{- 1} \binom{n
  - m}{m - c} \zeta_{c} (\eta) - \frac{m^{2}}{n} \zeta_{1} (\eta).
  \label{eqn--L2-dist-proj-as-comb-sum}
\end{equation}

\begin{theorem}
\label{thm--L2-dist-proj}
For \(m \in \mathbb{N} \setminus \{1\}\), let \(\eta\) be a
\(m\)\textsuperscript{th} order kernel that is permutation symmetric.
Furthermore, let \(F\) be a probability measure such that
\(\eta \in \mathscr{L}_{2} \left( F^{m} \right)\).
Let \(\rho_{n} (\eta, F)\) be as defined in \eqref{eqn--L2-dist-proj}.
In the case of \(m = 2\), we have the equality
\begin{equation}
  \rho_{n} (\eta, F) = \frac{2}{n (n - 1)} \left( \mathrm{E}_{F} \left[
  \mathrm{Var}_{F^{2}} \left[ \eta \left( Z_{1}, Z_{2} \right) \middle| Z_{1}
  \right] \right] - \mathrm{Var}_{F} \left[ \mathrm{E}_{F^{2}} \left[ \eta
  \left( Z_{1}, Z_{2} \right) \middle| Z_{1} \right] \right] \right).
  \label{eqn--L2-dist-proj-2}
\end{equation}
More generally, for any \(m \geq 2\), we can bound \(\rho_{n} (\eta, F)\)
by
\begin{equation}
  \begin{split}
  \rho_{n} (\eta, F) \leq
  & \, \frac{m^{2} (m - 1)^{2} \left( 1 + \frac{m - 1}{n - m + 1}
  \right)}{n^{2}} \zeta_{1} (\eta, F) \\
  & + \sum_{c = 2}^{m} \frac{1}{\gamma_{n, c}} \cdot \frac{(m !)^{2}}{((m -
c)!)^{2} c!}
  \left( 1 + \delta_{n, m, c} \right) \zeta_{c} (\eta, F),
  \end{split}
  \label{eqn--L2-dist-proj-m}
\end{equation}
where for each \(c \in \{2, \dots, m\}\), the quantities \(\delta_{n, m,
c}\) are do not depend on \(\eta\), \(F\) or the dimension of \(Z_{i}\) (except
possibly through \(n\) and \(m\)) and satisfy \(\lim_{n \to \infty} \delta_{n,
m, c}
= 0\).
\end{theorem}

\subsection{Proof of \texorpdfstring{\Cref{thm--L2-dist-proj}}{Theorem
\ref{thm--L2-dist-proj}} in the case of \texorpdfstring{\(m = 2\)}{m = 2}}

For \(m = 2\), \eqref{eqn--L2-dist-proj-as-comb-sum} becomes
\begin{align*}
  \rho_{n} (\eta) =
  & \, \frac{4}{n} \left( \frac{(n - 2)!}{(n - 1)!} \cdot \frac{(n - 2)!}{(n
  - 3)!} - 1 \right) \zeta_{1} (\eta) + \binom{n}{2}^{- 1} \zeta_{2} (\eta) \\
  =
  & \, \frac{4}{n} \left( \frac{n - 2}{n - 1} - 1 \right) \zeta_{1} (\eta)
  + \frac{2}{n (n - 1)} \zeta_{2} (\eta),
\end{align*}
and so
\begin{equation}
  \rho_{n} (\eta) =
  \frac{2}{n (n - 1)} \left( \zeta_{2} (\eta) - 2 \zeta_{1} (\eta) \right).
  \label{eqn--L2-dist-proj-2-init}
\end{equation}
Since \(m = 2\), \(\Pi_{2} \left( \cdot; \eta \right) = \eta (\cdot)\).
Using \eqref{eqn--zeta-c},
\begin{align*}
  \zeta_{2} (\eta) =
  & \, \mathrm{Var} \left[ \eta \left( Z_{1}, Z_{2} \right) \right] \\
  \text{[by Law of Total Variance]} \quad =
  & \, \mathrm{Var} \left[ \mathrm{E} \left[ \eta \left( Z_{1}, Z_{2} \right)
  \middle| Z_{1} \right] \right] + \mathrm{E} \left[ \mathrm{Var} \left[ \eta
  \left( Z_{1}, Z_{2} \right) \middle| Z_{1} \right] \right] \\
  =
  & \, \mathrm{Var} \left[ \Pi_{1} \left( Z_{1}; \eta \right) \right] +
  \mathrm{E} \left[ \mathrm{Var} \left[ \eta \left( Z_{1}, Z_{2} \right)
  \middle| Z_{1} \right] \right] \\
  =
  & \, \zeta_{1} (\eta) + \mathrm{E} \left[ \mathrm{Var} \left[ \eta \left(
  Z_{1}, Z_{2} \right) \middle| Z_{1} \right] \right].
\end{align*}
Then, from \eqref{eqn--L2-dist-proj-2-init}, using the fact that
\(\zeta_{1} (\eta) = \mathrm{Var} \left[ \mathrm{E} \left[ \eta \left( Z_{1},
Z_{2} \right) \middle| Z_{1} \right] \right]\) (from \eqref{eqn--zeta-c}),
\begin{equation*}
  \rho_{n} (\eta) = \frac{2}{n (n - 1)} \left( \mathrm{E} \left[
  \mathrm{Var} \left[ \eta \left( Z_{1}, Z_{2} \right) \middle| Z_{1} \right]
  \right] - \mathrm{Var} \left[ \mathrm{E} \left[ \eta \left( Z_{1}, Z_{2}
  \right) \middle| Z_{1} \right] \right] \right),
\end{equation*}
which is exactly \eqref{eqn--L2-dist-proj-2}.
\qed

\subsection{Proof of \texorpdfstring{\Cref{thm--L2-dist-proj}}{Theorem
\ref{thm--L2-dist-proj}} in the case of \texorpdfstring{\(m > 2\)}{m > 2}}

For \(m > 2\), recall that in \eqref{eqn--L2-dist-proj-as-comb-sum},
\begin{equation*}
  \rho_{n} (\eta) = \sum_{c = 1}^{m} \binom{m}{c} \binom{n}{m}^{- 1} \binom{n
  - m}{m - c} \zeta_{c} (\eta) - \frac{m^{2}}{n} \zeta_{1} (\eta).
\end{equation*}
For \(c = m\),
\begin{align*}
  \binom{n}{m}^{- 1} \binom{n - m}{0} =
  & \,
  \frac{(n - m)! m!}{n!} = \frac{m!}{\gamma_{n, m}}
\end{align*}
Thus,
\begin{equation*}
  \begin{split}
    \text{for } c = m, \quad
    \binom{n}{m}^{- 1} \binom{n - m}{m - c} =
    & \, \frac{m!}{(m - c)!} \frac{1}{\gamma_{n, c}} \left( 1 + \delta_{n, m, c}
    \right), \\
    \text{where } \delta_{n, m, c} = & \, 0, \text{ for each } n.
  \end{split}
\end{equation*}

Now let \(c \in \{2, \dots, m - 1\}\).
Then,
\begin{align*}
  \binom{n}{m}^{- 1} \binom{n - m}{m - c} =
  & \, \frac{(n - m)! m!}{n!} \cdot \frac{(n - m)!}{(n - 2 m + c)! (m - c)!} \\
  =
  & \, \frac{m!}{(m - c)!} \cdot \frac{(n - m)!}{n!} \frac{(n - m)!}{(n - 2 m +
  c)!}
  \\
  =
  & \, \frac{m!}{(m - c)!} \cdot \frac{(n - m) \cdots (n - 2 m + c + 1)}{n
  \cdots (n - m + 1)}.
\end{align*}
The numerator has \(m - c\) terms, whereas the denominator has \(m\) terms.
Hence for fixed \(m\) we can factor out \(1 / \gamma_{n, c}\) to get
\begin{align*}
  \binom{n}{m}^{- 1} \binom{n - m}{m - c} =
  & \, \frac{m!}{(m - c)!} \cdot \frac{(n - m) \cdots (n - 2 m + c + 1)}{n
  (n - 1) \cdots (n - m + 1)} \\
  =
  & \, \frac{m!}{(m - c)!} \cdot \frac{1}{n \cdots (n - c + 1)} \cdot \frac{(n -
  m) \cdots (n - 2 m + c + 1)}{(n - c) \cdots (n - m + 1)} \\
  =
  & \, \frac{m!}{(m - c)!} \cdot \frac{1}{\gamma_{n, c}} \cdot \frac{(n - m)
  \cdots (n - 2 m + c + 1)}{(n - c) \cdots (n - m + 1)}.
\end{align*}
Factorising the ratio in the end further,
\begin{equation*}
  \binom{n}{m}^{- 1} \binom{n - m}{m - c} =
  \frac{m!}{(m - c)!} \cdot \frac{1}{\gamma_{n, c}} \cdot \prod_{l = 0}^{m - c -
  1} \left( 1 - \frac{m - c}{n - c - l}  \right).
\end{equation*}
Therefore, set
\begin{equation}
  \delta_{n, m, c} = \left[ \prod_{l = 0}^{m - c - 1} \left( 1 - \frac{m - c}{n
  - c - l} \right) \right] - 1.
  \label{eqn--delta-nmc}
\end{equation}
Then,
\begin{equation*}
  \begin{split}
    \binom{n}{m}^{- 1} \binom{n - m}{m - c} =
    & \, \frac{m!}{(m - c)!} \frac{1}{\gamma_{n, c}} \left( 1 + \delta_{n, m, c}
    \right), \\
    \text{where } \lim_{n \to \infty} \delta_{n, m, c} =
    & \, 0, \text{ for each } c
    \leq m.
  \end{split}
\end{equation*}
Therefore, applying the above expression to
\eqref{eqn--L2-dist-proj-as-comb-sum}
\begin{align}
  \rho_{n} (\eta) =
  & \, \sum_{c = 1}^{m} \frac{1}{\gamma_{n, c}} \cdot \frac{(m
  !)^{2}}{((m - c)!)^{2} c!} \left( 1 + \delta_{n, m, c} \right) \zeta_{c}
  (\eta) - \frac{m^{2}}{n} \zeta_{1} (\eta) \nonumber \\
  =
  & \, \frac{m^{2}}{n} \delta_{n, m, 1} \zeta_{1} (\eta) + \sum_{c = 2}^{m}
  \frac{1}{\gamma_{n, c}} \cdot \frac{(m !)^{2}}{((m - c)!)^{2} c!} \left( 1 +
  \delta_{n, m, c} \right) \zeta_{c} (\eta).
  \label{eqn--L2-dist-proj-m-1}
\end{align}

For fixed \(\eta\), visual inspection of \eqref{eqn--L2-dist-proj-m-1}, first
suggests that the term corresponding to \(c = 1\) will exhibit the slowest
tendency towards zero, unless \(\delta_{n, m, 1} = O (1 / n)\).
We can show that \(\delta_{n, m, 1} = O (1 / n)\) is in fact true.
To that end, from \eqref{eqn--delta-nmc},
\begin{equation*}
  \delta_{n, m, 1} = \left( 1 - \frac{m - 1}{n - 1} \right) \cdots \left( 1 -
  \frac{m - 1}{n - m + 1}  \right) - 1.
\end{equation*}
By \citet[Lemma 1, p. 358]{1995billingsleyProbabilityMeasure},
given any \(l \in \mathbb{N}\) and \(z_{1, 1}, z_{2, 1}, \dots, z_{1, l}, z_{2,
l} \in \mathbb{C}\) with \(\left| z_{j, k} \right| \leq 1\),
\begin{equation*}
  \left| z_{1, 1} \cdots z_{1, l} - z_{2, 1} \cdots z_{2, l} \right| \leq
  \sum_{k = 1}^{l} \left| z_{1, k} - z_{2, k} \right|.
\end{equation*}
Applying this with \(l = m - 1\), \(z_{1, j} = 1 - ((m - 1) / (n - j))\) and
\(z_{2, j} = 1\) for every \(j = 1, \dots, m - 1\),
\begin{equation}
  \left| \delta_{n, m, 1} \right| \leq \sum_{k = 1}^{m - 1} \frac{m - 1}{n - k}
  \leq (m - 1)^{2} \left( 1 + \frac{m - 1}{n - m + 1} \right) \cdot \frac{1}{n}.
\end{equation}

Revisiting \eqref{eqn--L2-dist-proj-m-1},
\begin{align*}
  \rho_{n} (\eta) =
  & \, \frac{m^{2}}{n} \delta_{n, m, 1} \zeta_{1} (\eta) + \sum_{c = 2}^{m}
  \frac{1}{\gamma_{n, c}} \cdot \frac{(m !)^{2}}{((m - c)!)^{2} c!} \left( 1 +
  \delta_{n, m, c} \right) \zeta_{c} (\eta) \\
  \leq
  & \, \frac{m^{2}}{n} \left| \delta_{n, m, 1} \right| \zeta_{1} (\eta) +
  \sum_{c = 2}^{m} \frac{1}{\gamma_{n, c}} \cdot \frac{(m !)^{2}}{((m - c)!)^{2}
  c!} \left( 1 + \delta_{n, m, c} \right) \zeta_{c} (\eta),
\end{align*}
so that
\begin{equation*}
  \begin{split}
  \rho_{n} (\eta) \leq
  & \, \frac{m^{2} (m - 1)^{2} \cdot \left( 1 + \frac{m - 1}{n - m + 1}
  \right)}{n^{2}} \zeta_{1} (\eta) \\
  & + \sum_{c = 2}^{m} \frac{1}{\gamma_{n, c}} \cdot \frac{(m !)^{2}}{((m -
  c)!)^{2} c!} \left( 1 + \delta_{n, m, c} \right) \zeta_{c} (\eta),
  \end{split}
\end{equation*}
which is exactly \eqref{eqn--L2-dist-proj-m}.
\qed

\section{Asymptotic comparison of U- and V-statistics}

\begin{theorem}
\label{thm--u-v-asymp-compare}
If \(q \in [1, \infty)\) and \(\max_{i \in \mathbb{N}_{n}^{m}} \mathrm{E} \left[
\left\| \eta \left( Z_{i (1)}, \dots, Z_{i (m)} \right) \right\|^{q} \right] <
\infty\),
\begin{equation}
  \mathrm{E} \left[ \left\| U_{n} (\eta) - V_{n} (\eta) \right\|^{q}
  \right]^{\frac{1}{q}} \leq \frac{m (m - 1)}{n} \max_{i \in \mathbb{N}_{n}^{m}}
  \mathrm{E} \left[ \left\| \eta \left( Z_{i (1)}, \dots, Z_{i (m)} \right)
  \right\|^{q} \right]^{\frac{1}{q}}.
  \label{eqn--u-v-asymp-compare}
\end{equation}
\end{theorem}

\begin{proof}[Proof of \Cref{thm--u-v-asymp-compare}]
Apply \Cref{lem--VUW-comparison-fundamental}:
\begin{equation*}
  \mathrm{E} \left[ \left\| U_{n} (\eta) - V_{n} (\eta) \right\|^{q}
  \right]^{\frac{1}{q}} \leq \frac{2 \left( n^{m} - \gamma_{n, m}
  \right)}{n^{m}}
  \max_{i \in \mathbb{N}_{n}^{m}}
  \mathrm{E} \left[ \left\| \eta \left( Z_{i (1)}, \dots, Z_{i (m)} \right)
  \right\|^{q} \right]^{\frac{1}{q}}.
\end{equation*}
By \Cref{lem--nm-gammanm}, \(n^{m} - \gamma_{n, m} = (m (m - 1) / 2) n^{m - 1}\)
so that \(\left( 2 \left( n^{m} - \gamma_{n, m} \right) \right) / n^{m} = (m (m
- 1)) / n\).
Hence \eqref{eqn--u-v-asymp-compare} follows.
\end{proof}

\begin{lemma}
\label{lem--VUW-comparison-fundamental}
Let \(\mathcal{I}\) be a non-empty finite set and \(\left\{ a_{\iota} : \iota
\in \mathcal{I} \right\}\) be a subset of a normed space with norm
\(\|\cdot\|\).
Let \(\mathcal{I}_{0} \subseteq \mathcal{I}\) be a strict subset of
\(\mathcal{I}\), i.e. \(\left| \mathcal{I} \setminus \mathcal{I}_{0} \right|
\geq 1\).
Define
\begin{equation}
  V = \frac{1}{|\mathcal{I}|} \sum_{\iota \in \mathcal{I}} a_{\iota}, \qquad U =
  \frac{1}{\left| \mathcal{I}_{0} \right|} \sum_{\iota \in \mathcal{I}_{0}}
  a_{\iota}, \qquad W = \frac{1}{|\mathcal{I}| - \left| \mathcal{I}_{0} \right|}
  \sum_{\iota \in \mathcal{I} \setminus \mathcal{I}_{0}} a_{\iota}.
  \label{eqn--VUW-comparison-fundamental-VUW-defs}
\end{equation}
Then, the following hold
\begin{align}
  V - U =
  & \, \frac{|\mathcal{I}| - \left| \mathcal{I}_{0} \right|}{|\mathcal{I}|} (W
  - U),
  \label{eqn--VUW-comparison-fundamental-main}
  \\
  \|V - U\| \leq
  & \, 2 \frac{|\mathcal{I}| - \left| \mathcal{I}_{0} \right|}{|\mathcal{I}|}
  \max_{\iota \in \mathcal{I}} \left\| a_{\iota} \right\|.
  \label{eqn--VUW-comparison-fundamental-norm-bound}
\end{align}
\end{lemma}

\begin{proof}[Proof of \Cref{lem--VUW-comparison-fundamental}]
From \eqref{eqn--VUW-comparison-fundamental-VUW-defs},
\(|\mathcal{I}| V = \sum_{\iota \in \mathcal{I}_{0}} a_{\iota} + \sum_{\iota \in
\mathcal{I} \setminus \mathcal{I}_{0}} a_{\iota} = \left| \mathcal{I}_{0}
\right| U + \left( |\mathcal{I}| - \left| \mathcal{I}_{0} \right| \right) W\).
Subtract \(|\mathcal{I}| U\) from both sides, so that \(|\mathcal{I}| (V - U) =
\left( |\mathcal{I}| - \left| \mathcal{I}_{0} \right| \right) (W - U)\).
Division by \(|\mathcal{I}|\) establishes
\eqref{eqn--VUW-comparison-fundamental-main}.
For \eqref{eqn--VUW-comparison-fundamental-norm-bound}, by
\eqref{eqn--VUW-comparison-fundamental-main} and the triangle inequality,
\begin{align*}
  \|U - V\| =
  & \, \frac{|\mathcal{I}| - \left| \mathcal{I}_{0}
  \right|}{|\mathcal{I}|} \|U - W\| \leq \frac{|\mathcal{I}| - \left|
  \mathcal{I}_{0} \right|}{|\mathcal{I}|} (\|U\| + \|W\|) \\
  \leq
  & \, 2 \frac{|\mathcal{I}| - \left| \mathcal{I}_{0} \right|}{|\mathcal{I}|}
  \max \{\|U\|, \|W\|\}.
\end{align*}
Since \(U\) and \(W\) are arithmetic averages over subsets of \(\mathcal{I}\),
we get \(\max \{\|U\|, \|W\|\} \leq \max_{\iota \in \mathcal{I}} \left\|
a_{\iota} \right\|\) from repeated applications of the triangle inequality.
This establishes \eqref{eqn--VUW-comparison-fundamental-norm-bound}.
\end{proof}

\begin{lemma}
\label{lem--nm-gammanm}
For \(n, m \in \mathbb{N}\), with \(n \geq m\), with \(\gamma_{n, m}\) as in
\eqref{eqn--gammanm},
\begin{equation}
  n^{m} - \gamma_{n, m} = \frac{m (m - 1)}{2} n^{m - 1}.
  \label{eqn--nm-gammanm}
\end{equation}
\end{lemma}

\begin{proof}[Proof of \Cref{lem--nm-gammanm}]
For \(m = 1\), \eqref{eqn--nm-gammanm} follows from \(\gamma_{n, 1} = n\) by
definition in \eqref{eqn--gammanm}.
For \(m = 2\), by definition in \eqref{eqn--gammanm},
\(\gamma_{n, 2} = n (n - 1)\).
Thus
\begin{equation*}
  n^{2} - \gamma_{n, 2} = n^{2} - n (n - 1) = n,
\end{equation*}
so that \eqref{eqn--nm-gammanm} follows for \(m = 2\).

Now we verify \eqref{eqn--nm-gammanm} for \(m > 2\).
Note that
\begin{equation*}
  n^{m} - \gamma_{n, m} = \left| \mathrm{NonInj}_{n, m} \right|,
\end{equation*}
where as in \eqref{eqn--finite-domain-codomain-func-sets},
\begin{equation*}
  \mathrm{NonInj}_{n, m} = \left\{ i \in \mathbb{N}_{n}^{m} : \exists j, k
  \in \mathbb{N}_{m}, j \neq k \text{ such that } i (j) = i
  (k) \right\}.
\end{equation*}

Now consider the task of picking an element from \(\mathrm{NonInj}_{n, m}\).
Recall that we are considering the cases \(m > 2\).
We can exhaust \(\mathrm{NonInj}_{n, m}\) by using the following steps:
\begin{enumerate}
  \item Pick two indices \(j, k \in \mathbb{N}_{m}\) such that \(j
    < k\);
  \item Select \(i (j) \in \mathbb{N}_{n}\) and set \(i (k) = i
    (j)\);
  \item Pick the restriction \(\left\{ i (l) : l \in \mathbb{N}_{m} \setminus
    \{j, k\} \right\} \in \mathbb{N}_{n}^{m - 2}\) arbitrarily.
\end{enumerate}
There are \(m (m - 1) / 2\) ways to complete step 1.
For each instance of step 1, there are \(n\) ways to complete step 2.
Finally, for each instance of steps 1 and 2, there are \(n^{m - 2}\) ways to
complete step 3.
Therefore,
\begin{equation*}
  n - \gamma_{n, m} = \left| \mathrm{NonInj}_{n, m} \right| = \frac{m (m -
  1)}{2} \cdot n^{m - 1}.
\end{equation*}
\end{proof}

\printbibliography

\newpage

\appendix

\section{Preliminaries on projections}

Denote the set of random variables that are square-integrable against \(\Pr\) by
\begin{equation}
  \mathscr{L}_{2} \equiv \mathscr{L}_{2} (\Pr) = \left\{ X : \mathrm{E} \left[
  |X|^{2} \right] = \int |X|^{2} \mathrm{d} \Pr < \infty \right\}.
\end{equation}
It is well known that \(\mathscr{L}_{2}\) is a Hilbert space under the
inner product \(\langle X, Y \rangle = \mathrm{E} [X \cdot Y]\).
That is, the inner product space \(\left( \mathscr{L}_{2}, \langle \cdot, \cdot
\rangle \right)\) is a complete metric space under the induced normed metric:
\begin{equation}
  \rho_{2} \left( X, Y \right)^{2} := \|X - Y\|_{2}^{2} := \langle X - Y, X - Y
  \rangle = \mathrm{E} \left[ |X - Y|^{2} \right] \quad \forall \ X, Y \in
  \mathscr{L}_{2}.
\end{equation}

Let \(\mathcal{S} \subseteq \mathscr{L}_{2} (\Pr)\) and \(T \in
\mathscr{L}_{2}\).
\(T_{\ast} \in \mathcal{S}\) is a projection of \(T\) onto \(\mathcal{S}\) if
and only if
\begin{equation}
  T_{\ast} \in \mathcal{S} \quad \text{and} \quad \mathrm{E} \left[ \left( T -
  T_{\ast} \right)^{2} \right] \leq \mathrm{E} \left[ (T - S)^{2} \right] \quad
  \forall S \in \mathcal{S}.
  \label{eqn--proj-def}
\end{equation}

\subsection{Existence of projections onto closed and convex subsets}

\Cref{thm--proj-exists-closed-convex} shows that a projection on \(\mathcal{S}
\subset \mathscr{L}_{2}\) exists whenever \(\mathcal{S}\) is closed and convex.

\begin{theorem}
\label{thm--proj-exists-closed-convex}
Let \(T \in \mathscr{L}_{2}\) and let \(\mathcal{S} \subseteq \mathscr{L}_{2}\)
be closed and convex.
Then there exists \(T_{\ast}\) satisfying \eqref{eqn--proj-def}.
Furthermore, \(T_{\ast}\) is almost surely unique, i.e. if \(T_{\ast \ast}\)
also satisfies \eqref{eqn--proj-def}, then \(\Pr \left\{ T_{\ast} \neq T_{\ast
\ast} \right\} = 0\).
\end{theorem}

\begin{proof}[Proof of \Cref{thm--proj-exists-closed-convex}]
Let
\begin{equation*}
  d_{\ast} := \inf_{S \in \mathcal{S}} \mathrm{E} \left[ (S - T)^{2} \right].
\end{equation*}
Then for each \(n \in \mathbb{N}\), there necessarily exists \(S_{n} \in
\mathcal{S}\) such that
\begin{equation*}
  d_{\ast} \leq \mathrm{E} \left[ \left( S_{n} - T \right)^{2} \right] \leq
  d_{\ast} + \frac{1}{n}.
\end{equation*}
Furthermore for any \(n, m \in \mathbb{N}\),
\begin{align*}
  \mathrm{E} \left[ \left( S_{n} - S_{m} \right)^{2} \right] =
  & \, \mathrm{E} \left[ \left( S_{n} - T \right)^{2} \right] + \mathrm{E}
  \left[ \left( S_{m} - T \right)^{2} \right] - 2 \mathrm{E} \left[ \left( S_{n}
  - T \right) \left( S_{m} - T \right) \right], \\
  4 \mathrm{E} \left[ \left( \frac{S_{n} + S_{m}}{2} - T \right)^{2} \right] =
  & \, \mathrm{E} \left[ \left( S_{n} - T \right)^{2} \right] + \mathrm{E}
  \left[ \left( S_{m} - T \right)^{2} \right] + 2 \mathrm{E} \left[ \left( S_{n}
  - T \right) \left( S_{m} - T \right) \right].
\end{align*}
This implies
\begin{equation*}
  \mathrm{E} \left[ \left( S_{n} - S_{m} \right)^{2} \right] =
  2 \mathrm{E} \left[ \left( S_{n} - T \right)^{2} \right] + 2 \mathrm{E}
  \left[ \left( S_{m} - T \right)^{2} \right] - 4 \mathrm{E} \left[ \left(
  \frac{S_{n} + S_{m}}{2} - T \right)^{2} \right].
\end{equation*}
Since \(\mathcal{S}\) is convex, \(\frac{1}{2} \left( S_{n} + S_{m} \right) \in
\mathcal{S}\) and so, \(\mathrm{E} \left[ \left\{ \frac{1}{2} \left( S_{n} +
S_{m} \right) - T \right\}^{2} \right] \geq d_{\ast}\).
Using the upper bounds in the definition of \(S_{n}\) and \(S_{m}\), and the
lower bound now established,
\begin{equation*}
  \mathrm{E} \left[ \left( S_{n} - S_{m} \right)^{2} \right] \leq
  2 \left( d_{\ast} + \frac{1}{n} \right) + 2 \left( d_{\ast} + \frac{1}{m}
  \right) - 4 d_{\ast} = 2 \left( \frac{1}{n} + \frac{1}{m} \right).
\end{equation*}
Hence, \(\left\{ S_{n} \right\}\) is Cauchy in \(\mathcal{S}\).
Since \(\mathscr{L}_{2}\) is complete and \(\mathcal{S}\) is a closed subset of
\(\mathscr{L}_{2}\), \(\mathcal{S}\) is also necessarily complete.
Thus, there is a \(T_{\ast} \in \mathcal{S}\) such that
\(\lim_{n \to \infty} \mathrm{E} \left[ \left( S_{n} - T_{\ast} \right)^{2}
\right] = 0\).
Then,
\begin{align*}
  d_{\ast} \leq
  & \, \mathrm{E} \left[ \left( T_{\ast} - T \right)^{2} \right]
  =
  \mathrm{E} \left[ \left( S_{n} - T \right)^{2} \right] + \mathrm{E} \left[
  \left( T_{\ast} - S_{n} \right)^{2} \right] + 2 \mathrm{E} \left[ \left( S_{n}
  - T \right) \cdot \left( T_{\ast} - S_{n} \right) \right] \\
  \leq
  & \, \mathrm{E} \left[ \left( S_{n} - T \right)^{2} \right] + \mathrm{E}
  \left[ \left( T_{\ast} - S_{n} \right)^{2} \right] + 2 \mathrm{E} \left[
  \left( S_{n} - T \right)^{2} \right]^{\frac{1}{2}} \mathrm{E} \left[ \left(
  T_{\ast} - S_{n} \right)^{2} \right]^{\frac{1}{2}}.
\end{align*}
Hence,
\begin{equation*}
  d_{\ast} \leq \mathrm{E} \left[ \left( T_{\ast} - T \right)^{2} \right] \leq
  d_{\ast} + \frac{1}{n} + \mathrm{E} \left[ \left( T_{\ast}
  - S_{n} \right)^{2} \right] + 2 \left( d_{\ast} + \frac{1}{n}
  \right)^{\frac{1}{2}} \mathrm{E} \left[ \left( T_{\ast} - S_{n} \right)^{2}
  \right]^{\frac{1}{2}}.
\end{equation*}
Since \(\mathrm{E} \left[ \left( T_{\ast} - S_{n} \right)^{2} \right] \to 0\),
taking limits,
\(\mathrm{E} \left[ \left( T_{\ast} - T \right)^{2} \right] = d_{\ast} = \inf_{S
\in \mathcal{S}} \mathrm{E} \left[ \left( S - T \right)^{2} \right]\).

Next suppose \(T_{\ast \ast}\) also satisfies \eqref{eqn--proj-def}.
Then as before,
\begin{equation*}
  \mathrm{E} \left[ \left( T_{\ast} - T_{\ast \ast} \right)^{2} \right] =
  2 \mathrm{E} \left[ \left( T_{\ast} - T \right)^{2} \right] + 2 \mathrm{E}
  \left[ \left( T_{\ast \ast} - T \right)^{2} \right] - 4 \mathrm{E} \left[
  \left( \frac{T_{\ast} + T_{\ast \ast}}{2} - T \right)^{2} \right].
\end{equation*}
Since \(T_{\ast}\) and \(T_{\ast \ast}\) both satisfy \eqref{eqn--proj-def},
\begin{equation*}
  \mathrm{E} \left[ \left( T_{\ast} - T_{\ast \ast} \right)^{2} \right] =
  4 d_{\ast} - 4 \mathrm{E} \left[ \left( \frac{T_{\ast} + T_{\ast \ast}}{2} - T
  \right)^{2} \right].
\end{equation*}
Since \(\mathcal{S}\) is convex, \(\frac{T_{\ast} + T_{\ast \ast}}{2} \in
\mathcal{S}\) and so,
\begin{equation*}
  \mathrm{E} \left[ \left( \frac{T_{\ast} + T_{\ast \ast}}{2} - T
  \right)^{2} \right] \geq d_{\ast}.
\end{equation*}
Hence,
\begin{equation*}
  0 \leq \mathrm{E} \left[ \left( T_{\ast} - T_{\ast \ast} \right)^{2} \right]
  \leq 4 d_{\ast} - 4 d_{\ast} = 0.
\end{equation*}
The above implies that \(\Pr \left\{T_{\ast} \neq T_{\ast \ast} \right\} = 0\).
\end{proof}

\subsection{Projection onto linear subspaces and orthogonality}

When the subset for projection is a linear subspace of \(\mathscr{L}_{2}\),
a useful equivalent characterization of a projection is through an orthogonality
condition.

\begin{theorem}
\label{thm--proj-orth-cond}
Let \(T \in \mathscr{L}_{2}\) and \(\mathcal{S}\) be a linear subspace of
\(\mathscr{L}_{2}\).
Then \(T_{\ast}\) satisfies \eqref{eqn--proj-def} if and only if
\begin{equation}
  T_{\ast} \in \mathcal{S} \quad \text{and} \quad \mathrm{E} \left[ S \cdot
  \left( T - T_{\ast} \right) \right] = 0 \quad \forall S \in \mathcal{S}.
  \label{eqn--orthogonality-condition}
\end{equation}
Furthermore, \(T_{\ast}\) satisfying \eqref{eqn--orthogonality-condition} is
almost surely unique.
That is, if \(T_{\ast \ast}\) also satisfies
\eqref{eqn--orthogonality-condition}, then \(\Pr \left\{ T_{\ast} \neq T_{\ast
\ast} \right\} = 0\).
\end{theorem}

\begin{proof}[Proof of \Cref{thm--proj-orth-cond}]
Suppose \(T_{\ast}\) satisfies \eqref{eqn--orthogonality-condition}.
Then, for any \(S \in \mathcal{S}\),
\begin{align*}
  \mathrm{E} \left[ (T - S)^{2} \right] =
  & \, \mathrm{E} \left[ \left( T - T_{\ast} \right)^{2} \right] + 2 \mathrm{E}
  \left[ \left( T - T_{\ast} \right) \cdot \left( T_{\ast} - S \right) \right] +
  \mathrm{E} \left[ \left( T_{\ast} - S \right)^{2} \right] \\
  =
  & \, \mathrm{E} \left[ \left( T - T_{\ast} \right)^{2} \right] + \mathrm{E}
  \left[ \left( T_{\ast} - S \right)^{2} \right],
\end{align*}
where the final equality follows from \eqref{eqn--orthogonality-condition} since
\(T_{\ast} - S \in \mathcal{S}\) by linearity of \(\mathcal{S}\).
Therefore, since \(\mathrm{E} \left[ \left( T_{\ast} - S \right)^{2} \right]
\geq 0\), \(T_{\ast}\) satisfies \eqref{eqn--proj-def}.

Next, we prove the converse by proving its contrapositive.
To that end, suppose that \eqref{eqn--orthogonality-condition} fails.
That is, \(T_{\ast} \not\in \mathcal{S}\) or there exists \(S_{\ast} \in
\mathcal{S}\) such that
\begin{equation}
  \mathrm{E} \left[ S_{\ast} \cdot \left( T - T_{\ast} \right) \right] \neq 0.
  \label{eqn--orthogonality-condition-fail}
\end{equation}
If \(T_{\ast} \not\in \mathcal{S}\), then it is immediate that
\eqref{eqn--proj-def} fails.
Thus, suppose that \(T_{\ast}, S_{\ast} \in \mathcal{S}\) and
\eqref{eqn--orthogonality-condition-fail} holds.
Note that \eqref{eqn--orthogonality-condition-fail} necessitates \(\Pr \left\{
S_{\ast} \neq 0 \right\} > 0\), i.e. \(S_{\ast}\) cannot be almost surely zero.
Since \(\mathcal{S} \subseteq \mathscr{L}_{2}\), this also implies that \(0 <
\mathrm{E} \left[ S_{\ast}^{2} \right] < \infty\).
By linearity of \(\mathcal{S}\), \(T_{\ast} + \tau S_{\ast} \in \mathcal{S}\)
for every \(\tau \in \mathbb{R}\), and
\begin{equation*}
  \mathrm{E} \left[ \left( T - T_{\ast} - \tau S_{\ast} \right)^{2} \right] =
  \mathrm{E} \left[ \left( T - T_{\ast} \right)^{2} \right] - 2 \tau \mathrm{E}
  \left[ S_{\ast} \cdot \left( T - T_{\ast} \right) \right] + \tau^{2}
  \mathrm{E} \left[ S_{\ast}^{2} \right].
\end{equation*}
As a function of \(\tau\), the above is minimized at
\begin{equation*}
  \tau_{\ast} = \frac{\mathrm{E} \left[ S_{\ast} \cdot \left( T - T_{\ast}
  \right) \right]}{ \mathrm{E} \left[ S_{\ast}^{2} \right]} \neq 0.
\end{equation*}
The associated minimum value is
\begin{align*}
  \mathrm{E} \left[ \left( T - T_{\ast} - \tau_{\ast} S_{\ast} \right)^{2}
  \right] =
  & \, \mathrm{E} \left[ \left( T - T_{\ast} \right)^{2} \right] -
  \frac{\mathrm{E} \left[ S_{\ast} \cdot \left( T - T_{\ast} \right)
  \right]^{2}}{\mathrm{E} \left[ S_{\ast}^{2} \right]} < \mathrm{E} \left[
  \left( T - T_{\ast} \right)^{2} \right].
\end{align*}
This means that \eqref{eqn--proj-def} fails.
By contrapositive, we have that if \eqref{eqn--proj-def} holds, then
\eqref{eqn--orthogonality-condition} must also hold.

Finally, to see that \(T_{\ast}\) is almost surely unique, let \(T_{\ast \ast}\)
also satisfy \eqref{eqn--orthogonality-condition}.
Then,
\begin{equation*}
  \mathrm{E} \left[ \left( T_{\ast} - T_{\ast \ast} \right)^{2} \right] =
  \mathrm{E} \left[ \left( T_{\ast} - T_{\ast \ast} \right) \cdot \left(
  T_{\ast} - T \right) \right] + \mathrm{E} \left[ \left( T_{\ast} - T_{\ast
  \ast} \right) \cdot \left( T - T_{\ast \ast} \right) \right] = 0,
\end{equation*}
since by linearity of \(\mathcal{S}\), \(T_{\ast} - T_{\ast \ast} \in
\mathcal{S}\).
The above implies that \(\Pr \left\{T_{\ast} \neq T_{\ast \ast} \right\} = 0\).
\end{proof}

\subsection{Second-moment distance between a random variable and its projection}

The following result provides a useful characterization of the second
moment distance between a random variable and its projection onto a space that
contains the constants.

\begin{theorem}
\label{thm--proj-L2-dist-as-dif-of-vars}
Let \(T \in \mathscr{L}_{2}\) and \(\mathcal{S}\) be a
linear subspace of \(\mathscr{L}_{2}\) containing the constants
(i.e. non-stochastic random variables).
Suppose the projection of \(T\) onto \(\mathcal{S}\) exists and denote this
projection by \(T_{\ast}\).
Then,
\begin{equation}
  \mathrm{E} \left[ \left( T - T_{\ast} \right)^{2} \right] =
  \mathrm{Var} [T] - \mathrm{Var} \left[ T_{\ast} \right].
  \label{eqn--proj-L2-dist-as-dif-of-vars}
\end{equation}
\end{theorem}

\begin{proof}[Proof of \Cref{thm--proj-L2-dist-as-dif-of-vars}]
By \(T_{\ast} \in \mathcal{S}\) and \eqref{eqn--orthogonality-condition},
\begin{equation}
  \mathrm{E} \left[ T_{\ast} \cdot \left( T - T_{\ast} \right) \right] = 0
   \iff \mathrm{E} \left[ T \cdot T_{\ast} \right] = \mathrm{E} \left[
   T_{\ast}^{2} \right].
  \label{eqn--prod-mom-equal-proj-2-mom}
\end{equation}
Similarly, since \(\mathcal{S}\) contains the constants,
\begin{equation}
  \mathrm{E} \left[ T - T_{\ast} \right] = 0 \iff \mathrm{E} [T] =
  \mathrm{E} \left[ T_{\ast} \right].
  \label{eqn--means-equal-proj}
\end{equation}
Therefore, by \eqref{eqn--prod-mom-equal-proj-2-mom} and
\eqref{eqn--means-equal-proj},
\begin{equation*}
  \mathrm{Cov} \left( T, T_{\ast} \right) = \mathrm{E} \left[ T T_{\ast}
  \right] - \mathrm{E} [T] \mathrm{E} \left[ T_{\ast} \right] = \mathrm{E}
  \left[ T_{\ast}^{2} \right] - \mathrm{E} \left[ T_{\ast} \right]^{2}.
\end{equation*}
Hence,
\begin{equation}
  \mathrm{Cov} \left( T, T_{\ast} \right) = \mathrm{Var} \left[ T_{\ast}
  \right].
  \label{eqn--covar-TThat-equal-var-That}
\end{equation}

Next, by \eqref{eqn--means-equal-proj}, \(\mathrm{E} \left[ T - T_{\ast} \right]
= 0\) and so,
\begin{equation*}
  \mathrm{E} \left[ \left( T - T_{\ast} \right)^{2} \right] = \mathrm{Var}
  \left[ T - T_{\ast} \right] = \mathrm{Var} [T] + \mathrm{Var} \left[
  T_{\ast} \right] - 2 \mathrm{Cov} \left( T, T_{\ast} \right).
\end{equation*}
Therefore by \eqref{eqn--covar-TThat-equal-var-That},
\begin{equation}
  \mathrm{E} \left[ \left( T - T_{\ast} \right)^{2} \right] = \mathrm{Var}
  [T] - \mathrm{Var} \left[ T_{\ast} \right],
\end{equation}
which is exactly \eqref{eqn--proj-L2-dist-as-dif-of-vars}.
\end{proof}

\subsection{Projection onto sum spaces}

Suppose now that \(\left\{ Z_{j} \right\}_{j = 1}^{n}\) are independent random
variables/vectors and let
\begin{equation}
  \mathcal{S} = \left\{ \sum_{j = 1}^{n} g_{j} \left( Z_{j} \right) : g_{j}
  \text{ measurable and } \mathrm{E} \left[ g_{j} \left( Z_{j} \right)^{2}
  \right] < \infty \text{ for all } j = 1, \dots, n \right\}.
  \label{eqn--sum-space-1}
\end{equation}

\begin{theorem}
\label{thm--proj-sum-space}
Let \(\left\{ Z_{j} \right\}_{j = 1}^{n}\) be independent random variables.
The projection of an arbitrary random variable \(T \in \mathscr{L}_{2}\) onto
the class \(\mathcal{S}\) in \eqref{eqn--sum-space-1} is given by
\begin{equation*}
  T_{\ast} = \sum_{j = 1}^{n} \mathrm{E} \left[ T \middle| Z_{j} \right] - (n -
  1) \mathrm{E} [T] = \mathrm{E} [T] + \sum_{j = 1}^{n} \left\{ \mathrm{E}
  \left[ T \middle| Z_{j} \right] - \mathrm{E} [T] \right\}.
  % \label{eqn--proj-sum-space}
\end{equation*}
\end{theorem}

\begin{proof}[Proof of \Cref{thm--proj-sum-space}]
We proceed by verifying the orthogonality condition
\eqref{eqn--orthogonality-condition}.
Let \(j \in \{1, \dots, n\}\) be given.
If \(g_{j} (\cdot)\) is a measurable function with \(\mathrm{E} \left[ g_{j}
\left( Z_{j} \right)^{2} \right] < \infty\), then certainly \(g_{j} \left( Z_{j}
\right) \in \mathcal{S}\).
Furthermore,
\begin{align*}
  \mathrm{E} \left[ g_{j} \left( Z_{j} \right) \cdot \left( T - T_{\ast} \right)
  \right] =
  & \, \mathrm{E} \left[ g_{j} \left( Z_{j} \right) \cdot \left\{ T - \mathrm{E}
  [T] - \mathrm{E} \left[ T \middle| Z_{j} \right] + \mathrm{E} [T] \right\}
  \right] \\
  & - \sum_{\substack{l = 1 \\ l \neq j}}^{n} \mathrm{E} \left[ g_{j} \left(
  Z_{j} \right) \cdot \left\{ \mathrm{E} \left[ T \middle| Z_{l} \right] -
  \mathrm{E} [T] \right\} \right] \\
  =
  & \, \mathrm{E} \left[ g_{j} \left( Z_{j} \right) \cdot \left\{ T - \mathrm{E}
  \left[ T \middle| Z_{j} \right] \right\}
  \right] \\
  [\text{by } Z_{j} \indep Z_{l}] \qquad & - \sum_{\substack{l = 1 \\ i
  \neq j}}^{n} \mathrm{E} \left[ g_{j} \left( Z_{j} \right) \right] \cdot
  \mathrm{E} \left[ \left\{ \mathrm{E} \left[ T \middle| Z_{l} \right] -
  \mathrm{E} [T] \right\} \right] \\
  =
  & \, 0.
\end{align*}
In the last line above, the second term is zero since \(\mathrm{E} \left[
\mathrm{E} \left[ T \middle| Z_{l} \right] \right] = \mathrm{E} [T]\) for every
\(l\) and the first term is zero by definition of \(\mathrm{E} \left[ T \middle|
Z_{j} \right]\).
Now, take a sum of such functions \(g_{j} \left( Z_{j} \right)\) over \(j = 1,
\dots, n\):
\begin{equation*}
  \mathrm{E} \left[ \left( \sum_{j = 1}^{n} g_{j} \left( Z_{j} \right) \right)
  \cdot \left( T - T_{\ast} \right) \right] = \sum_{j = 1}^{n} \mathrm{E} \left[
  g_{j} \left( Z_{j} \right) \cdot \left( T - T_{\ast} \right) \right] = \sum_{j
  = 1}^{n} 0 = 0.
\end{equation*}
\end{proof}

\section{Exercises in counting}

\begin{theorem}
\label{thm--picking-with-common-elements}
Let \(m, p, n\) be natural numbers and \(c\) be a non-negative integer such
that \(c \leq \min \{m, p\}\) and \(\max \{m, p\} \leq n\).
Let \(a = \left\{ a_{i} \right\}_{i = 1}^{m}, b = \left\{ b_{i} \right\}_{i =
1}^{p} \subseteq \{1, \dots, n\}\) where \(i \mapsto a_{i}\) and \(i \mapsto
b_{i}\) are both strictly increasing.
There are \(\binom{n}{m} \cdot \binom{m}{c} \cdot \binom{n - m}{p - c}\)
ways to select \(a\) and \(b\) such that they have exactly \(c\) elements in
common.
\end{theorem}

\begin{proof}[Proof of \Cref{thm--picking-with-common-elements}]
Let \(C\) denote the number of ways to pick \(a\) and \(b\) with exactly \(c\)
common elements.
We can select \(a\) and \(b\) with exactly \(c\) common elements by first
selecting \(a\), then selecting \(c\) elements of \(a\) that will be common with
\(b\), and finally selecting \(b\) to ensure only those \(c\) elements are
common to \(a\) and \(b\).
Therefore,
\begin{equation*}
  \begin{split}
    C =
    & \, \# \text{ of ways to select } a \\
    & \times \# \text{ of ways to select } c \text{ elements of } a \text{ in
    common with } b \\
    & \times \# \text{ of ways to select remaining elements of } b \\
  \end{split}
\end{equation*}
Clearly,
\begin{align*}
  \# \text{ of ways to select } a =
  & \, \binom{n}{m}, \\
  \# \text{ of ways to select } c \text{ elements of } a =
  & \, \binom{m}{c}.
\end{align*}
Then, consider the last problem of selecting the remaining elements of \(b\)
that are not common with \(a\).
Since the \(c\) common elements have already been selected, there are \(p - c\)
remaining elements of \(b\) requiring selection.
Since we have fulfilled the ``exactly \(c\) elements in common''
constraint, this selection must be from \(\{1, \dots, n\} \setminus a\), which
has cardinality \(n - m\).
Therefore, upon fixing \(a\) and the \(c\) common elements,
\begin{equation*}
  \# \text{ of ways to select remaining elements of } b = \binom{n - m}{p - c}.
\end{equation*}
\end{proof}

\end{document}
